\documentclass[a4paper,11pt]{article}
\usepackage[T1]{fontenc}
\usepackage[utf8]{inputenc}
\usepackage[english,bulgarian]{babel}
\usepackage{lmodern}
\usepackage{tabularx}
\usepackage{pgf,tikz}
\usetikzlibrary{arrows}

\title{Планини}

\begin{document}
\pagestyle{empty}
\maketitle
{\large
Дадени са $n$ на брой планини с височини \\

\begin{equation}
    a [0], a [1], ... , a [n - 1]
\end{equation}\\
За тях е изпълнено, че съществува $k$, такова че \\
\begin{center}
\definecolor{qqwuqq}{rgb}{1.,0.,0.}
\begin{tikzpicture}[line cap=round,line join=round,>=triangle 45,x=0.5cm,y=0.25cm]
\draw (0,1)--(1,5);
\draw (1,5)--(2,6);
\draw (2,6)--(3,7);
\draw (3,7)--(4,10);
\draw (4,10)--(5,8);
\draw (5,8)--(6,5);
\draw (6,5)--(7,3);
\draw (7,3)--(8,6);
\draw (8,6)--(9,5);
\draw[color=black,fill=black,fill opacity=1](0,1) circle (0.25);
\draw[color=black,fill=black,fill opacity=1](1,5) circle (0.25);
\draw[color=black,fill=black,fill opacity=1](2,6) circle (0.25);
\draw[color=black,fill=black,fill opacity=1](3,7) circle (0.25);
\draw[color=qqwuqq,fill=qqwuqq,fill opacity=1](4,10) circle (0.25);
\draw[color=black,fill=black,fill opacity=1](5,8) circle (0.25);
\draw[color=black,fill=black,fill opacity=1](6,5) circle (0.25);
\draw[color=black,fill=black,fill opacity=1](7,3) circle (0.25);
\draw[color=black,fill=black,fill opacity=1](8,6) circle (0.25);
\draw[color=black,fill=black,fill opacity=1](9,5) circle (0.25);
\end{tikzpicture}
\end{center}
\begin{equation}
    a [0] < a [1] < ... < a [k - 1] < a [k] > a [k + 1] > ... > a [n - 1]
\end{equation}\\ \vspace{0.25cm}
Напишете програма \textbf{\otherlanguage{english}{mountains}}, която  по размери на планините отпечатва на стандартния изход числото $k$. \\ \\
\textbf{\otherlanguage{english}{Input:}} \\
На първия ред на стандартния вход се въвежда едно цяло число $n$.\\
На втория ред на стандартния вход се въвеждат $n$ цели числа - \\
$a [0], a [1], ... , a [n - 1]$ (в този ред). \\ \\
\textbf{\otherlanguage{english}{Output:}} \\
На първия ред на стандартния изход изведете едно цяло число $k$.\\
}
\begin{center}

{\large
\begin{table}[ht]
    \begin{tabular}{l|l}
        \textbf{\otherlanguage{english}{Input:}} & \textbf{\otherlanguage{english}{Output:}} \\
        10     & 4       \\
        1 5 6 7 10 8 5 3 6 5 & \\
    \end{tabular}
\end{table}
}
\end{center}

\end{document}
